%%%%%%%%%%%%%%%%%%%%%%%%%%%%%%%%%%%%%%%
%% Makros & anderer Low-Level bastel %%
%%%%%%%%%%%%%%%%%%%%%%%%%%%%%%%%%%%%%%%
\makeatletter
%% Makros für Titel, Autor und Datum 
%% Dank diesem Makro stehen Titel, Autor und Datum überall im Dokument zur verfügung
%% Date hat zudem den Default-Wert \today
\def\@Title{}
\def\@Author{}
\def\@Date{\today}
\newcommand{\Title}{\@Title}
\newcommand{\Author}{\@Author}
\newcommand{\Date}{\@Date}
\AtBeginDocument{%
  \let\@Title\@title
  \let\@Author\@author
  \let\@Date\@date
}

%% Makros für den Arraystretch (bei uns meist in Tabellen genutzt, welche Formeln enthalten)
% Default Value
\def\@ArrayStretchDefault{1} % Entspricht der Voreinstellung von Latex

% Setzt einen neuen Wert für den arraystretch
\newcommand{\setArrayStretch}[1]{\renewcommand{\arraystretch}{#1}}

% Setzt den arraystretch zurück auf den default wert
\newcommand{\resetArrayStretch}{\renewcommand{\arraystretch}{\@ArrayStretchDefault}}

% Makro zum setzten des Default arraystretch. Kann nur in der Präambel verwendet werden.
\newcommand{\setDefaultArrayStretch}[1]{%
	\AtBeginDocument{%
		\def\@ArrayStretchDefault{#1}
		\renewcommand{\arraystretch}{#1}
	}
}
\makeatother


%%%%%%%%%%%%%%%%%%%%%%%
%% Wichtige Packages %%
%%%%%%%%%%%%%%%%%%%%%%%
\usepackage[utf8]{inputenc} % UTF-8 unterstützung
\usepackage[english, ngerman]{babel} % Silbentrennung
\usepackage[automark]{scrpage2} % Header und Footer
\usepackage{tabularx}

% Für Abbildungen mit mehreren kleinen Bilder
% Doku: http://www.ctan.org/tex-archive/macros/latex/contrib/caption/
\usepackage{caption}
\usepackage{subcaption}

% Für die Nutzung von "tabular" (Tabulator)
\usepackage{listliketab}
\usepackage{placeins}