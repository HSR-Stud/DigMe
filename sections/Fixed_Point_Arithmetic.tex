\section{Fixed Point Arithmetic}
\subsection{Einleitung}$~$ \\
\begin{tabular}{| l | l | l |}
\hline
& Unsigned integer & Signed integer\\
\hline
Formel & $z=\sum_{k=0}^{n-1}a_k \cdot 2^k$ & $z=-(2^{n-1}) \cdot a_{n-1} + \sum_{k=0}^{n-2}2^k \cdot a_k$\\
\hline
Bereich & $0 \ldots 2^n-2^0$ & $-2^{n-1} \ldots 2^{n-1}-2^0$\\
\hline
\end{tabular}

\subsection{Qn.m Format}$~$ \\
n: Anzahl integer bits (inkl. eines allfälligen Vorzeichenbits); m: Anzahl der Nachkommastellen\\
Es gibt zwei Möglichkeiten, eine Fix-point Zahl in VHDL zu implementieren:
\begin{compactitem}
  \item Nur mit dem 'numeric\_std'-Package und dabie unsigned und signed Typen zu verwenden. Der Programmierer muss praktisch alles selbst machen (z.B. Dezimalpunkt tracken)
  \item Mithilfe des 'fixed\_pkg-Packages' (basiert auf 'numeric\_std') $\rightarrow$ einfacherer, übersichtlicherer Code; fast alle Operatoren sind für integer und reelle Typen überladen; Overflows sind nicht möglich $\rightarrow$ führt immer zu sicherer Implementation, aber nicht immer zur optimalen Lösung in Bezug auf die Vektorbreite
\end{compactitem}
Sizing-Regeln für 'fixed\_pkg':\\
\begin{tabular}{| l | l |}
\hline
Operation & Result Range\\
\hline
$A+B$ & Max(A'left,B'left)+1 downto Min(A'right,B'right)\\
\hline
$A-B$ & Max(A'left,B'left)+1 downto Min(A'right,B'right)\\
\hline
$A \cdot B$ & A'left+B'left+1 downto A'right+B'right\\
\hline
$A \text{ rem } B$ & Min(A'left,B'left)+1 downto Min(A'right,B'right)\\
\hline
$\text{Signed } /$ & A'left-B'right+1 downto A'right-B'left\\
\hline
$\text{Signed A mod B}$ & Min(A'left,B'left) downto Min(A'right,B'right)\\
\hline
$\text{Signed Reciprocal(A)}$ & -A'right downto -A'left-1\\
\hline
$\text{Abs(A)}$ & A'left+1 downto A'right\\
\hline
$-A$ & A'left+1 downto A'right\\
\hline
$\text{Unsigned} /$ & A'left-B'right downto A'right-B'left-1\\
\hline
$\text{Unsigned A mod B}$ & B'left downto Min(A'right,B'right)\\
\hline
$\text{Unsigned Reciprocal(A)}$ & -A'right+1 downto -A'left\\
\hline
\end{tabular}

\subsubsection{ufixed/sfixed}
Das Package definiert zwei neue Datentypen: ufixed und sfixed. Für negative zahlen wird das 2er-Komplement verwendet. Die Position des Dezimalpunkts ist zwischen Index ''0'' und ''-1''.
\lstinputlisting[language=VHDL]{code/ufixed_sfixed.vhd}

\subsubsection{Limitierte Unterstützung}
Das Package ist Teil von VHDL 2008, weshalb es von gewissen Tools noch nicht vollständig unterstützt wird. XILINX bietet eine modifizierte Version von 'fixed\_pkg' an.
% \lstinputlisting[language=VHDL]{code/fixed_pkg.vhd}

\subsubsection{Unsigned/Signed}
\begin{tabular}{| l | l | l |}
\hline
&  uQn.m & uQn.m\\
\hline
Formel & $z=\sum_{k=-m}^{n-1}a_k \cdot 2^k$ & $z=-(2^{n-1}) \cdot a_{n-1} + \sum_{k=-}^{n-2}a_k \cdot 2^k$\\
\hline
Bereich & $0 \ldots 2^n-2^{-m}$ & $-2^{n-1} \ldots 2^{n-1}-2^{-m}$\\
\hline
\end{tabular}\\
\textbf{Bsp.: uQ2.6}
\lstinputlisting[language=VHDL]{code/uQ2_6.vhd}
\textbf{Bsp.: sQ2.6}
\lstinputlisting[language=VHDL]{code/sQ2_6.vhd}

\subsection{Arithmetic Functions}
\subsubsection{Addition/Subtraction}
\paragraph{Unsigned and Unsigned}$~$ \\
\begin{minipage}{0.3\linewidth}
Praktisch identisch mit Add./Sub. mit normalen Integer-Zahlen, einzig muss nun der Programmierer den Dezimalpunkt tracken.
\end{minipage}
\begin{minipage}{0.7\linewidth}
\begin{tabular}{l l l l l l l l l | l}
& \textbf{0} & 1 & 0 & 1 & 1 & 0 & 1 & \textbf{0} & = 101.101 in uQ3.3 = 5.625\\
+ & & & & 1 & 0 & 0 & 1 & 1 & = 1.0011 in uQ1.4 = 1.1875\\
\hline
& \textbf{0} & 1 & 1 & 0 & 1 & 1 & 0 & 1 & = 0110.1101 in uQ4.4 = 6.8125\\
\end{tabular}
\end{minipage}
\lstinputlisting[language=VHDL]{code/addition_subtraktion.vhd}

\paragraph{Signed and Signed}$~$ \\
\begin{minipage}{0.3\linewidth}
Auch fast identisch, Zahl mit weniger Vorkommabits muss sign-extended werden
\end{minipage}
\begin{minipage}{0.7\linewidth}
  \begin{tabular}{l l l l l l l l l | l}
  & \textbf{1} & 1 & 0 & 1 & 1 & 0 & 1 & \textbf{0} & = 101.101 in sQ3.3 = -2.375\\
  + & \textbf{1} & \textbf{1} & \textbf{1} & 1 & 0 & 0 & 1 & 1 & = 1.0011 in sQ1.4 = -0.8125\\
  \hline
  & \textbf{1} & 1 & 0 & 0 & 1 & 1 & 0 & 1 & = 1100.1101 in sQ4.4 = -3.1875\\
  \end{tabular}
\end{minipage}
\lstinputlisting[language=VHDL]{code/addition_subtraktion_signed.vhd}

\subsubsection{Multiplication}
\paragraph{Unsigned by Unsigned}$~$ \\
Einfachster Fall, partielle Produkte werden ohne sign-extension addiert\\
\begin{minipage}{0.7\linewidth}
  \begin{tabular}{ l l l l l l l l | l}
  & & & & 1 & 1 & 0 & 1 & = 11.01 in uQ2.2 = 3.25 (multiplicand)\\
   & & & 1 & 0 & 1 & 1 & = 10.11 in uQ2.2 = 2.75 (multiplier)\\
  \hline
  & & & & 1 & 1 & 0 & 1 & \\
  & & & 1 & 1 & 0 & 1 & & \\
  & & 0 & 0 & 0 & 0 & & & \\
  & 1 & 1 & 0 & 1 & & & & \\
  \hline
  1 & 0 & 0 & 0 & 1 & 1 & 1 & 1 & = 1000.1111 in uQ4.4 = 8.9375
  \end{tabular}
\end{minipage}
\lstinputlisting[language=VHDL]{code/multiplikation_unsigned.vhd}

\paragraph{Signed by Unsigned}$~$ \\
Hier braucht jedes partielle Produkt sign-extension. In VHDL ist $\text{signed} \cdot \text{unsigned}$ nicht erlaubt. Der vorzeichenlose Wert muss zuerst in eine vorzeichenbehaftete Zahl gewandelt werden.\\
\begin{minipage}{0.7\linewidth}
  \begin{tabular}{ l l l l l l l l | l}
  & & & & 1 & 1 & 0 & 1 & = 11.01 in sQ2.2 = -0.75 (multiplicand)\\
  & & & & 0 & 1 & 0 & 1 & = 01.01 in sQ2.2 = 1.25 (multiplier)\\
  \hline
  \textbf{1} & \textbf{1} & \textbf{1} & \textbf{1} & 1 & 1 & 0 & 1 & \\
  \textbf{0} & \textbf{0} & \textbf{0} & 0 & 0 & 0 & 0 & & \\
  \textbf{1} & \textbf{1} & 1 & 1 & 0 & 1 & & & \\
  \textbf{0} & 0 & 0 & 0 & 0 & & & & \\
  \hline
  1 & 1 & 1 & 1 & 0 & 0 & 0 & 1 & = 1111.0001 in sQ4.4 = -0.9375
  \end{tabular}
\end{minipage}

\paragraph{Signed by Signed}$~$ \\
Zwei Möglichkeiten:
\begin{compactitem}
  \item Sign-extension aller partiellen Produkte. Wenn der 'multiplier' negativ ist, muss das letzte partielle Produkt mit dem 2er-Komplement transformiert werden.
  \item Beide Zahlen werden mit dem 2er-Komplement transformiert. Das partielle Produkt wird sign-extended, wenn der neue 'multiplicand' negativ ist.
\end{compactitem}
Das MSB des Produkts ist redundant. Wird das Produkt nach links geschoben, kann das redundante Bit entfernt und ein zusätzliches fractional bit hinzugefügt werden (neues Format: Q(n1+n2-1).(m1+m2+1)).\\
\begin{minipage}{0.7\linewidth}
  \begin{tabular}{ l l l l l l l l | l}
  & & & & 0 & 1 & 0 & 1 & = 01.01 in sQ2.2 = 1.25 (multiplicand)\\
  & & & & 1 & 0 & 1 & 1 & = 10.11 in sQ2.2 = -1.25 (multiplier)\\
  \hline
  \textbf{0} & \textbf{0} & \textbf{0} & \textbf{0} & 0 & 1 & 0 & 1 & \\
  \textbf{0} & \textbf{0} & \textbf{0} & 0 & 1 & 0 & 1 & & \\
  \textbf{0} & \textbf{0} & 0 & 0 & 0 & 0 & & & \\
  \textbf{1} & 1 & 0 & 1 & 1 & & & & 2's complement of multiplicand 0101\\
  \hline
  1 & 1 & 1 & 0 & 0 & 1 & 1 & 1 & 110.01110 in sQ3.5 = -1.5625\\
  & & & & & & & & (one of the sign bits is removed)\\
  \end{tabular}
\end{minipage}\\
\begin{minipage}{0.7\linewidth}
  \begin{tabular}{ l l l l l l l l | l}
  & & & & 1 & 0 & 1 & 1 & = 2's complement of multiplicand 0101\\
  & & & & 0 & 1 & 0 & 1 & = 2's complement of multiplier 1011\\
  \hline
  \textbf{1} & \textbf{1} & \textbf{1} & \textbf{1} & 1 & 0 & 1 & 1 & multiplicand is negative $\rightarrow$ extended sign-bits\\
  \textbf{0} & \textbf{0} & \textbf{0} & 0 & 0 & 0 & 0 & & \\
  \textbf{1} & \textbf{1} & 1 & 0 & 1 & 1 & & & \\
  \textbf{0} & 0 & 0 & 0 & 0 & & & & \\
  \hline
  1 & 1 & 1 & 0 & 0 & 1 & 1 & 1 & 110.01110 in sQ3.5 = -1.5625
  \end{tabular}
\end{minipage}\\.
\lstinputlisting[language=VHDL]{code/multiplikation_signed.vhd}
\textbf{Corner case:} Wenn die zwei kleinst möglichen Zahlen multipliziert werden (bei sQ2.2: 1000*1000). Dies führt zu einem Overflow und zu einem falschen Ergebnis (-4 anstatt +4). $\rightarrow$ \textbf{Lösung:} Das zweite Vorzeichenbit als Schutz stehen lassen. Dies ist so auch im 'fixed\_pkg'-Package umgesetzt

\subsubsection{Division}
Divisionen mit Mehrfachen von 2 ($2^n$) sind einfache Schiebeoperationen. Für andere Werte ist es komplizierter. 'fixed\_pkg' bietet eine elegante Lösung an:
\lstinputlisting[language=VHDL]{code/division.vhd}

\subsection{Overflow and Saturation}$~$ \\
Bei Verwendung des 'fixed\_pkg'-Packages werden Overflow und Saturation automatisch gelöst. Ansonsten gibt es zwei Lösungen mit einem Over-/Underflow umzugehen: Das Resultat sättigen (Maximum oder Minimum) oder das Resultat erhält ein zusätzliched integer bit.\\
Erkennung bei vorzeichenbehafteten Zahlen:
\begin{compactitem}
  \item Overflow, wenn Summe von zwei positiven Zahlen ein negatives Resultat liefert.
  \item Underflow, wenn Summe von zwei negativen Zahlen ein positives Resultat liefert.
  \item Die Summe einer positiven und negativen Zahl ergibt nie einen Underflow.
\end{compactitem}
Erkennung bei vorzeichenlosen Zahlen:
\begin{compactitem}
  \item Overflow, wenn Summe einer Addition kleiner als der erste Summand ist.
  \item Underflow, wenn Differenz einer Subtraktion grösser als der Subtrahend ist.
\end{compactitem}
\textbf{fixed\_pkg:} Der einzige Fall, dass ein Underflow entstehen kann, ist wenn das Resultat resized wird in eine kleiner Grösse. Die resize-Funktion von 'numeric\_std' wird überladen:
\lstinputlisting[language=VHDL]{code/resize_fixed_pkg.vhd}
Mit der nun vorhandenen resize-Funktion ist es möglich, das Overflow-Handling, 'truncation' oder 'rounding' in einem Schritt durchzuführen. Der Default 'overflow\_style' ist 'fixed\_saturate', kann aber auch auf 'fixed\_wrap' zu setzen um einen Overflow zu ermöglichen. Der Default 'round\_style' ist 'fixed\_round' um das Resultat zu runden, kann aber auch auf 'fixed\_truncate' gesetzt werden.

\subsection{Scaling}$~$ \\
Das Resultat einer Multiplikation hat die Breite des 'multiplicand' plus 'multiplier', mehrere aufeinanderfolgende Multiplikationen führen somit zu Bit Growth.

\subsubsection{Truncation}$~$ \\
Eine Möglichkeit: Bits mit tiefer Präzision weglassen. Die normale 'truncation' ist eine floor-Funktion. In VHDL kann das mit 'shift\_right' und 'resize' umgesetzt werden:
\lstinputlisting[language=VHDL]{code/truncation.vhd}
